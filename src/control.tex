Once a GP model, $\mathcal{M}_{\mathrm{P}}$ or $\mathcal{M}_{\mathrm{T}}$, of a building has been learnt, it can be used to develop a data-driven predictive controller for the building.
This section will describe the formulations of two GP-based building controllers.

In a predictive control algorithm, a process model, which is a GP model in our case, is used to predict the process output at multiple future time steps over a fixed horizon.
In such a multistep simulation of a dynamical GP, the autoregressive outputs fed to the model (recall $x_{t}$ in \eqref{E:GP:features}) beyond the first step are random variables, resulting in more and more complex output distributions as we go further.
Therefore, uncertainty propagation through the model should be accounted for, which would complicate the computation of the model significantly.
In \cite{nghiemetal16gp} we showed that a simple simulation method called \emph{zero-variance method}, which replaces the autoregressive output signals with their corresponding expected values, could achieve sufficient prediction accuracy while keeping the computation fairly simple and scalable.
We will use the zero-variance method in this paper.
Specifically, at time step $t$, the predicted output at a future time step $t+\tau$, for $\tau \geq 0$, is $y_{t+\tau} \sim \GaussianDist{\bar{y}_{t+\tau}}{\sigma^2_{y, t+\tau}}$ where
\begin{align}
  \label{eq:mean-variance-dynamics}
  \bar{y}_{t+\tau} =& \mu(x_{t+\tau}) + K_\star K^{-1} (Y - \mu(X)) \\
  \sigma^2_{y,t+\tau} =& K_{\star \star} - K_\star K^{-1} K_\star^T \nonumber \\
  x_{t + \tau} =& [\bar y_{t+ \tau-l}, \dots, \bar y_{t+ \tau-1}, u_{t+ \tau-m}, \dots, u_{t+ \tau}, \nonumber \\
                      & \quad d_{t+ \tau-p}, \dots, d_{t+ \tau-1}, d_{t+ \tau}]\text, \nonumber
\end{align}
in which $K$, $K_{\star}$, and $K_{\star\star}$ are defined in~\eqref{eq:gp-regression}.
Here, it is assumed that at time \(t\), all future disturbances \(d_{t+i}\) in the above $x_{t+\tau}$ are available from forecasts or fixed rules as applicable.
We also fix the horizon length to $N$ time steps.


\subsection{Demand Tracking Control}

This predictive control formulation fits demand response applications where a customer -- a building in this case -- must curtail its power demand by a certain amount from its baseline, or must track an Area Control Signal (ACS) sent from the grid operator.
For this formulation we only require GP model \(\mathcal{M}_{\mathrm{P}}\).
We are interested in solving the following finite-horizon control problem
\begin{align}
  &\minimize \sum_{\tau=0}^{N-1} (\bar{y}_{t+\tau} - y_{\mathrm{ref},t+\tau})^2 + \lambda \sigma^2_{y,t+\tau} \label{E:mpc:tracking} \\
  & 
    \begin{aligned}
      \text{subject to}\ \  & \text{Dynamic equations \eqref{eq:mean-variance-dynamics} of $\mathcal{M}_{\mathrm{P}}$} \\
      & u_{t+\tau} \in \mathcal{U} \nonumber 
    \end{aligned}
\end{align}
where the constraints hold for all \(\tau \in \{0,\dots,N-1\}\).
The signal $y_{\mathrm{ref}}$ is a reference power demand trajectory the building should follow as closely as possible.
The output variances $\sigma^{2}_{y}$ are included in the cost function to penalize highly uncertain predictions; in other words, we improve the control performance confidence by exploiting the capability of the GP model to provide predictive uncertainty measures.

We solve the optimization problem~\eqref{E:mpc:tracking} to compute optimal \(u_{t}^*, \dots, u_{t+N-1}^*\), apply \(u_{t}^*\) to the system, proceed to the next time step where we solve \eqref{E:mpc:tracking} at \(t+1\), and repeat.
Although all the constraints in \eqref{E:mpc:tracking} are of analytical forms, the optimization can be computationally hard to solve due to the high nonlinearity and computational burden of the GP.
% For the case study in Sec.~\ref{S:casestudy} we choose a combination of a squared exponential and rational quadratic kernel which results in nonconvex problem.
We use the nonlinear solver IPOPT \cite{Waechter2009b} and the optimization modeling framework CasADi \cite{Andersson2013b} to solve \eqref{E:mpc:tracking}.
\note{May remove one or both citations (or do not mention which solver and tool we used.}

\subsection{Climate Control with Minimum Energy Usage}


This formulation aims to minimize the building's energy usage while meeting thermal comfort and operation constaints. 
This is always desired since in practice the consumers want to minimize their electricity bills but maintain acceptable thermal comfort for building occupants.
We will require both GP models \(\mathcal{M}_{\mathrm{P}}\) and \(\mathcal{M}_{\mathrm{T}}\).
Specifically, \(\mathcal{M}_{\mathrm{T}}\) is used to enforce thermal comfort constraints in the optimization problem below:
\begin{subequations}
  \label{E:mpc:climate}
  \begin{align}
    \minimize \ & \sum_{\tau=0}^{N-1} \bar{y}_{t+\tau} + \lambda \sigma^2_{y,t+\tau} \\
    \text{subject to}\ \  & \text{Dynamic equations \eqref{eq:mean-variance-dynamics} of $\mathcal{M}_{\mathrm{P}}$} \\
                &\bar{T}_{t+\tau} = \mu(x_{t+\tau}) + K_\star K^{-1} (Y - \mu(X)) \label{E:mpc:climate:tempmean} \\
                &\sigma^2_{T,t+\tau} = K_{\star \star} - K_\star K^{-1} K_\star^T \label{E:mpc:climate:tempvar} \\
                &\Pr(T_{t+\tau} \in \mathcal{T}) \geq 1 - \epsilon \label{E:mpc:climate:chance} \\
                &u_{t+\tau} \in \mathcal{U} 
  \end{align}
\end{subequations}
Here, \(\bar{T},\sigma^2_{T}\) denote the mean and variance of the temperature output returned by \(\mathcal{M}_{\mathrm{T}}\). 
The constraints~\eqref{E:mpc:climate:tempmean} and \eqref{E:mpc:climate:tempvar} describe the zone temperature dynamics by the GP model $\mathcal{M}_{\mathrm{T}}$.
The constraint~\eqref{E:mpc:climate:chance} is a chance constraint, which states that the zone temperature must stay inside a set $\mathcal{T}$ with a probability of at least $1 - \epsilon$.
Our goal is to find the control inputs \(u\) that minimize the power consumption while maintaining thermal comfort with high probability. 
We again solve the optimization problem~\eqref{E:mpc:climate} to compute optimal \(u_{t}^*, \dots, u_{t+N-1}^*\), apply \(u_{t}^*\) to the system and repeat at time \(t+1\).



%%% Local Variables:
%%% mode: latex
%%% TeX-master: "main"
%%% End:
