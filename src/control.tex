\begin{todo}
  Using GP in MPC. Briefly present various applications (as in BPACS paper).
  Adapt the following text.
\end{todo}

Using these measurement data, we learn autoregressive GP models \(\mathcal{M}_1\) and \(\mathcal{M}_2\), and use the zero-variance method to predict the future output \(y_{t+\tau}\), where $t$ is the current time and \( \tau \ge 0\).
Specifically,
\begin{gather}
\label{eq:dpc:prediction}
y_{t+\tau} \sim \GaussianDist{\bar{y}_{t+\tau} = g_{\mathrm m}(x_{t+\tau})}{\sigma^2_{y, t+\tau} = g_{\mathrm v}(x_{t+\tau})}, \\
x_{t + \tau} = [\bar y_{t+ \tau-l}, \dots, \bar y_{t+ \tau-1}, u_{t+ \tau-m}, \dots, u_{t+ \tau}, \nonumber \\
\qquad\qquad\qquad\qquad  w_{t+ \tau-p}, \dots, w_{t+ \tau-1}, w_{t+ \tau}]\text, \nonumber
\end{gather}
in which \(w:=[d^w, d^p]\).
It is assumed that at time \(t\), \(w_{t+\tau}\) are available \(\forall \tau \) from forecasts or fixed rules as applicable.

We optimize the hyperparameters \(\theta\) % = [\mu, k, \sigma_{f_2}, \lambda_d, \sigma_{f_3}, \alpha, \lambda] \)
of the GP model using GPML \cite{Rasmussen2010}.


In a multistep simulation of a dynamical GP, the autoregressive outputs fed to the model beyond the first step are random variables, resulting in more and more complex output distributions as we go further.
Therefore, it involves uncertainty propagation through the model, which would complicate the computation of the model significantly.
\cite{nghiemetal16gp} showed that a simple simulation method called \emph{zero-variance method}, which replaces the autoregressive signals with their corresponding expected values, could achieve sufficient prediction accuracy while benefitting from computational simplicity.
In this paper, the zero-variance method was selected for predicting future outputs in optimization formulations.


%%% Local Variables:
%%% mode: latex
%%% TeX-master: "main"
%%% End:
